\chapter{Combinatorial Surfaces}
\label{chap:combo_surfaces}



\begin{definition}
  \label{def:fin_abstr_simpl_complex}
  \lean{fin_abstr_simpl_complex}
  \leanok
  An \emph{(finite) abstract simplicial complex} is a pair $(V,\K)$, 
  where $V$ is a finite set, 
  $\K \subseteq \P(V)$ is a set of subsets of $V$, 
  every $\sigma \in \K$ is finite, 
  and for all $\sigma \in \K$, 
  $\sigma' \subseteq \sigma$ implies $\sigma' \in \K$. 
  $V$ is called the set of \emph{vertices} 
  and elements of $\K$ are called \emph{simplices}.
\end{definition}

\begin{definition}
  \label{def:fin_abstr_simplex}
  \lean{fin_abstr_simplex}
  \leanok
  Let $(V,\K)$ be a finite simplicial complex. 
  For $k \in \N$, an \emph{abstract $k$-simplex} is 
  a simplex $\sigma \in \K$ consisting of exactly $k+1$ vertices.
\end{definition}
