\chapter{Combinatorial Surfaces}
\label{chap:combinatorial_surfaces}

\section{Abstract simplicial complexes}

\begin{definition}\label{def:abstract_simplicial_complex}
    \leanok
    \lean{abstract_simplicial_complex}
    An \emph{abstract simplicial complex} $\K$ is a set of non-empty\footnote{Some authors allow simplices to be empty.} finite sets called \emph{simplices} such that
    for all $\sigma \in \K$, 
    if $\sigma' \subseteq \sigma$ 
    and $\sigma' \neq \emptyset$,
    then $\sigma' \in \K$.
\begin{comment}
    An \emph{abstract simplicial complex} is a pair $(V,\K)$, 
    where $V$ is a set, 
    $\K \subseteq \Pow(V)$ is a set of finite %, non-empty 
    subsets of $V$, 
    %every $\sigma \in \K$ is finite, 
    and for all $\sigma \in \K$, 
    if $\sigma' \subseteq \sigma$ 
    %and $\sigma' \neq \emptyset$, 
    then $\sigma' \in \K$. 
    $V$ is called the set of \emph{vertices} 
    and elements of $\K$ are called \emph{simplices}.
\end{comment}
\end{definition}

Abstract simplicial complexes are the combinatorial or topological analogs of \emph{geometric simplicial complexes}, which we will see shortly. Abstract simplicial complexes capture the connectivity of geometric simplicial complexes without their geometry.

\begin{comment}
\begin{definition}\label{def:abstract_simplex}
    \uses{def:abstract_simplicial_complex}
    \leanok
    \lean{abstract_simplicial_complex.abstract_simplex}
    Let $\K$ %$(V,\K)$ 
    be a finite simplicial complex. 
    For $k \in \N$, an \emph{abstract $k$-simplex} is 
    a simplex $\sigma \in \K$ consisting of exactly $k+1$ vertices.
\end{definition}
\end{comment}

\begin{definition}\label{def:degree}
    \uses{def:abstract_simplicial_complex}
    %\lean{abstract_simplicial_complex.degree}
    The \emph{degree} or \emph{dimension} of an (abstract) simplex with $k+1$ vertices is $k$. 
\end{definition}

\begin{definition}\label{def:k_simplices}
    \uses{def:degree}
    An (abstract) simplex of degree $k$ is called an \emph{abstract $k$-simplex}.
\end{definition}
For instance, $0$-simplices are points or vertices, $1$-simplices are line segments or edges, $2$-simplices are triangles or faces, and $3$-simplices are tetrahedra. %Under the convention that it is a simplex, the empty set is the only $(-1)$-simplex.

\begin{definition}\label{def:face} % this might not need to be coded in lean
    \uses{def:abstract_simplicial_complex}
    A nonempty subset of an abstract simplex is called a \emph{face}.
\end{definition}
By definition of an abstract simplicial complex, all of the faces of a simplex are themselves simplices. A proper subset of a simplex is called a \emph{proper face}.

\begin{definition}\label{def:subcomplex}
    \uses{def:abstract_simplicial_complex}
    Given two simplicial complexes $\K$ and $\K'$, we say that $\K'$ is a \emph{subcomplex} of $\K$ if $\K' \subseteq \K$, that is, every simplex in $\K'$ is a simplex in $\K$ as well.
\end{definition}

\begin{definition}\label{def:abstract_simplicial_k-complex}
    \uses{def:abstract_simplicial_complex,def:degree}
    If the set of degrees of the simplices of an abstract simplicial complex has a maximum $k$, then that abstract simplicial complex is said to be an \emph{abstract simplicial $k$-complex}.
\end{definition}

\begin{definition}\label{def:pure_abstract_simplicial_k-complex}
    \uses{def:abstract_simplicial_k-complex,def:degree,def:face}
    Let $\K$ be an abstract simplicial $k$-complex. If every simplex is a face of some $k$-simplex, then we say $\K$ is \emph{pure} and call it a \emph{pure simplicial $k$-complex}.
\end{definition}

\begin{definition}\label{def:star}
    The \emph{star} $\St(\sigma)$ of a simplex $\sigma$ in an abstract simplicial complex $\K$ is the set of all simplices in $\K$ having $\sigma$ as a face.

    The star $\St(S)$ of a set $S$ of simplices in $\K$ (i.e., a subset of $\K$) is the set of all simplices in $\K$ having some simplex in $S$ as a face. Equivalently, $\St(S) = \bigcup_{\sigma \in S} \St(\sigma)$.
\end{definition}

\begin{definition}\label{def:closure}
    \uses{def:abstract_simplicial_complex}
    The \emph{closure} $\Cl(S)$ of a set of simplices $S \subseteq \K$ is the smallest (by inclusion) simplicial complex containing $S$.
\end{definition}

\begin{lemma}\label{lem:closure_eq_down_closure}
    \uses{def:closure}
    \[ \Cl(S) = \bigcup_{\sigma \in S} \Pow(\sigma) \setminus \emptyset. \] % idk why \setminus isn't showing up
\end{lemma}
%\begin{proof} \end{proof}

\begin{definition}\label{def:link}
    \uses{def:abstract_simplicial_complex}
    The \emph{link} $\Lk(\sigma)$ of a simplex $\sigma \in \K$ is the set of all simplices $\tau \in \K$ such that $\sigma$ and $\tau$ are disjoint and their union is a simplex. In set-builder notation,
    \[ \Lk(\sigma) \coloneqq \{ \tau \in \K \mid \tau \cap \sigma = \emptyset, \tau \cup \sigma \in \K \}. \]
\end{definition}

\begin{lemma}\label{lem:lk_eq_cl_st_sub_st_cl}
    \uses{def:link,def:closure,def:star}
    For any set of simplices $S \subseteq \K$, 
    \[ \Lk(S) = \Cl(\St(S)) \setminus \St(\Cl(S)). \]
\end{lemma}

\subsection{The halfedge mesh}
\label{ssec:halfedge_mesh}



\section{Simplicial complexes}
\label{sec:simplicial_complex}

