\chapter{Combinatorial Surfaces}
\label{chap:combinatorial_surfaces}



\begin{definition}
  \label{def:abstract_simplicial_complex}
  \lean{abstract_simplicial_complex}
  \leanok
  An \emph{abstract simplicial complex} is a pair $(V,\K)$, 
  where $V$ is a set, 
  $\K \subseteq \Pow(V)$ is a set of finite, non-empty subsets of $V$, 
  %every $\sigma \in \K$ is finite, 
  and for all $\sigma \in \K$, 
  if $\sigma' \subseteq \sigma$ and $\sigma' \neq \emptyset$, 
  then $\sigma' \in \K$. 
  $V$ is called the set of \emph{vertices} 
  and elements of $\K$ are called \emph{simplices}.
\end{definition}

\begin{definition}
  \label{def:abstract_simplex}
  \uses{def:abstract_simplicial_complex}
  \lean{abstract_simplicial_complex.abstract_simplex}
  \leanok
  Let $(V,\K)$ be a finite simplicial complex. 
  For $k \in \N$, an \emph{abstract $k$-simplex} is 
  a simplex $\sigma \in \K$ consisting of exactly $k+1$ vertices.
\end{definition}

\begin{definition}
    \label{def:degree}
    \uses{def:abstract_simplicial_complex}
    \lean{abstract_simplicial_complex.degree}
    The \emph{degree} of an abstract simplex of size $k$ is $k-1$. For instance, $0$-simplices are vertices, $1$-simplices are edges, $2$-simplices are faces, and $3$-simplices are tetrahedra. The empty set is the only $-1$-simplex.
\end{definition}

